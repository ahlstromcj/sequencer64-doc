%-------------------------------------------------------------------------------
% seq64_metrics
%-------------------------------------------------------------------------------
%
% \file        seq64_metrics.tex
% \library     Documents
% \author      Chris Ahlstrom
% \date        2015-11-10
% \update      2015-11-10
% \version     $Revision$
% \license     $XPC_GPL_LICENSE$
%
%     Provides a discussion of the MIDI and GUI metrics and how they work.
%
%-------------------------------------------------------------------------------

\section{Sequencer64 Metrics Issues}
\label{sec:metrics_issues}

   This section goes into the details of MIDI metrics and user-interface
   metrics.  \textsl{Seqeuencer64} is full of hardwired constants that
   affect both the playback of MIDI data the display of MIDI data.
   If we're going to be able to support variation in things like playback time
   and parts-per-quarter-notes per song, then we have to understand these
   constants perfectly and pin down all of the effects of modifying them.

   STILL INCOMPLETE AND IN PROGRESS.

\subsection{MIDI Metrics}
\label{subsec:metrics_issues_midi}

   TODO.

\subsubsection{MIDI Metrics, PPQN}
\label{subsubsec:metrics_issues_midi_ppqn}

   TODO.

\subsection{User-Interface Metrics}
\label{subsec:metrics_issues_ui}

   TODO.

\subsubsection{User-Interface Metrics, Sequence Editor}
\label{subsubsec:metrics_issues_ui_seqedit}

   The sequence or pattern editor (module \texttt{seqedit}) allows for the
   creations of a sequence with a particular time signature.  Each sequence can
   have its own time signature.

   TODO.

\subsubsection{User-Interface Metrics, Song Editor}
\label{subsubsec:metrics_issues_ui_perfedit}

   The song or performance editor (module \texttt{perfedit}) allows for the
   layout of a number of sequences, each of which can have its own
   particular time signature.
   Furthermore, the song editor has a piano roll grid that has its own, single
   time signature, which can be set to a number of different time signature
   values.  What are the visible effects of time signature on the appearance
   and layout of the sequences?  What internal \textsl{Sequencer64}
   variables control the appearance and layout?

   TODO.

   WE NEED TO MAKE A TABLE, KEEP THIS SAMPLE FOR NOW.

   \begin{table}
      \centering
      \caption{BOGUS}
      \label{table:BOGUS}
      \begin{tabular}{l l l l}
         \textbf{Application}  &
            \textbf{Legacy} &
            \textbf{New} & 
            \textbf{Original File} \\
         ardour       & TBD       & TBD       & TBD \\
         composite    & TBD       & TBD       & TBD \\
      \end{tabular}
   \end{table}

%-------------------------------------------------------------------------------
% vim: ts=3 sw=3 et ft=tex
%-------------------------------------------------------------------------------
