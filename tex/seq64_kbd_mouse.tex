%-------------------------------------------------------------------------------
% seq64_kbd_mouse
%-------------------------------------------------------------------------------
%
% \file        seq64_kbd_mouse.tex
% \library     Documents
% \author      Chris Ahlstrom
% \date        2016-04-07
% \update      2016-10-29
% \version     $Revision$
% \license     $XPC_GPL_LICENSE$
%
%     Provides tables for keyboard and mouse support in Sequencer64.
%
%-------------------------------------------------------------------------------

\section{Sequencer64 Keyboard and Mouse Actions}
\label{sec:kbd_mouse_actions}

   This section presents some tables summarizing the keyboard and mouse actions
   available in \textsl{Seqeuencer64}.
   It does not cover the mute keys and the groups keys, which are well
   described in the keyboard setup for the main window (where "live"
   performance is controlled;
   see \sectionref{paragraph:seq64_menu_file_options_keyboard}).
   It also does not cover the "fruity" mouse actions.  Any volunteers to fill
   in those tables?

   This section describes the keystrokes that are currently hardwired
   in \textsl{Sequencer64}.
   This description only includes
   items not defined in the \textbf{File / Options}
   dialog.  That is, hardwired values.
   Note that "KP" stands for "keypad".

\subsection{Main Window}
\label{subsec:kbd_mouse_main_window}

   The main window keystrokes are all defined via the options dialog
   and "rc" configuration file, or are stock Gtk window-management keystrokes.
   The main window has a very complete setup for live control of the MIDI tune
   via keystrokes.  These actions are not included in
   \tableref{table:main_window_support}.
   There may be some other keystrokes to be documented at some point.

   \begin{table}[H]
      \centering
      \caption{Main Window Support}
      \label{table:main_window_support}
      \begin{tabular}{l l l l l l}
         \textbf{Action}   & \textbf{Normal} & \textbf{Double}    & \textbf{Shift}      & \textbf{Ctrl}    & \textbf{Mod4}      \\
         e key             & ---             & ---                & ---                 & Open song editor & ---                \\
         Left-click slot   & Mute/Unmute     & New/Edit           & Toggle other slots  & Edit             & ---                \\
         Right-click slot  & Edit menu       & ---                & Edit menu           & Edit/Edit Menu   & ---                \\
      \end{tabular}
   \end{table}

   The new mouse features of this window for \textsl{Sequencer64},
   as noted in \sectionref{sec:seq64_patterns_panel}, are:

   \begin{itemize}
      \item \textsl{Shift-left-click}:
         Over one pattern slot, this action toggles the mute/unmute (armed/unarmed)
         status of all other patterns (even the patterns in other,
         unseen sets).
      \item \textsl{Left-double-click}:
         Over a pattern slot, this action quickly toggles the mute/unmute status,
         which is confusing.  But it ultimately brings up the pattern editor
         (sequence editor) for that pattern.  It acts like
         Ctrl-left-click.
   \end{itemize}

\subsection{Performance Editor Window}
\label{subsec:kbd_mouse_performance_editor_window}

   The "performance editor" window is also known as the "song editor" window.
   It's main sections are the "piano roll" (perfroll) and the "performance
   time" (perftime) sections, discussed in the following sections.
   Also, some keystrokes are handled by the frame of the window.

   \begin{itemize}
      \item \texttt{Ctrl-z}. Undo.
      \item \texttt{Ctrl-r}. Redo.
   \end{itemize}

\subsubsection{Performance Editor Piano Roll}
\label{subsubsec:kbd_mouse_performance_editor_piano_roll}

   \begin{itemize}
      \item \texttt{Ctrl-x}. Cut.
      \item \texttt{Ctrl-c}. Copy.
      \item \texttt{Ctrl-v}. Paste.
      \item \texttt{Ctrl-z}. Undo.
      \item \texttt{Ctrl-r}. Redo.
      \item \texttt{Shift-Up}.   Move backward one small unit (which is...?)
      \item \texttt{Shift-Down}.   Move forward one small unit (which is...?)
      \item \texttt{Shift-Page Up}.   Move backward one frame.
      \item \texttt{Shift-Page Down}.   Move forward one frame.
      \item \texttt{Shift-Home, Shift-KP Home}.  Move to beginning of piano roll.
      \item \texttt{Shift-End, Shift-KP End}.  Move to end of piano roll.
      \item \texttt{Shift-z (Z)}.  Zoom in.
      \item \texttt{0}.  Set default zoom.
      \item \texttt{z}.  Zoom out.
      \item \texttt{Left}.  Move item left one snap unit.
      \item \texttt{Right}.  Move item right one snap unit.
      \item \texttt{Up}.  Move frame up one small scroll unit.
      \item \texttt{Down}.  Move frame down one small scroll unit.
      \item \texttt{Home}.  Move to top of piano roll.
      \item \texttt{End}.  Move to bottom of piano roll.
      \item \texttt{Page Up}.  Move up one frame (page-increment).
      \item \texttt{Page Down}.  Move down one frame (page-increment).
   \end{itemize}

   Note that the keystrokes in this table
   (see \tableref{table:perf_window_piano_roll})
   require that the focus first be
   assigned to the piano roll by left-clicking in an empty area within it.
   Otherwise, another section of the performance editor might receive the
   keystroke.

   \begin{table}[H]
      \centering
      \caption{Performance Window Piano Roll}
      \label{table:perf_window_piano_roll}
      \begin{tabular}{l l l l l l}
         \textbf{Action}   & \textbf{Normal} & \textbf{Double}    & \textbf{Shift}     & \textbf{Ctrl}   & \textbf{Mod4}      \\
         Space             & Start playback  & ---                & ---                & ---             & ---                \\
         Esc               & Stop playback   & ---                & ---                & ---             & ---                \\
         Period (.)        & Pause playback  & ---                & ---                & ---             & ---                \\
         Del               & Cut section     & ---                & ---                & ---             & ---                \\
         c key             & ---             & ---                & ---                & Copy            & ---                \\
         p key             & Paint mode      & ---                & ---                & ---             & ---                \\
         v key             & ---             & ---                & ---                & Paste           & ---                \\
         x key             & Escape paint    & ---                & ---                & Cut             & ---                \\
         z key             & Zoom out        & ---                & ---                & Undo            & ---                \\
         0 key             & Reset zoom      & ---                & ---                & ---             & ---                \\
         Z key             & Zoom in         & ---                & ---                & Undo            & ---                \\
         Left-arrow        & Move earlier    & ---                & ---                & ---             & ---                \\
         Right-arrow       & Move later      & ---                & ---                & ---             & ---                \\
         Left-click        & Select section  & ---                & Toggle other slots & ---             & ---                \\
         Right-click       & Paint mode      & ---                & Paint mode         & Paint mode      & Lock Paint mode    \\
         Scroll-up         & Scroll up       & ---                & Scroll Left        & Scroll Up       & ---                \\
         Scroll-down       & Scroll down     & ---                & Scroll Right       & Scroll Down     & ---                \\
      \end{tabular}
   \end{table}

   This section of the performance editor also handles the start, stop, and
   pause keys.  These can be modified in the \textbf{Options / Keyboard} page.

   Note that a "section" in the performance editor is actually a box that
   specifies a trigger for the pattern in that sequence/pattern slot.

   Note that the "toggle other slots" action occurs only if shift-left-clicked
   in the "names" area of the performance editor.

   Note that left-click is used to select performance blocks if clicked within
   a block, or to deselect them if clicked in an empty area of the piano roll.
   Also note that all scrolling is done by the internal horizontal and vertical
   step increments.
   The new features of this window for \textsl{Sequencer64},
   as noted in \sectionref{sec:seq64_song_editor}, are:

   \begin{itemize}
      \item \textsl{p}:  Enters the paint mode, until right-click is pressed or
         until the "x" key is pressed.
      \item \textsl{x}:  Exits the paint mode.  Think of the made-up term
         "x-scape".
      \item \textsl{z}:  Zooms out the performance view.  It makes the view
         look smaller, so that more of the performance can be seen.  Please
         note that opening a second performance view is another way to see more
         of the performance.
      \item \textsl{0}:  Resets the performance view zoom to its normal value.
      \item \textsl{Z}:  Zooms in the performance view.  It makes the view
         look bigger, so that more details of the performance can be seen.
      \item \textsl{.}:  The period (configurable) is a new key devoted to the
         new pause functionality.
      \item \textsl{Left Arrow}:  Moves the selected item to the left (earlier
         in time) in the performance layout.
      \item \textsl{Right Arrow}:  Moves the selected item to the right (later
         in time) in the performance layout.
      \item \textsl{Mod4-right-click, release}:  Locks the paint mode, until right-click
         is pressed again later.
      \item Once selected (and thus rendered in grey), a pattern section (trigger)
         can be moved by the mouse.  To move it using the left or right
         arrow keys, the paint mode must be entered, but only via the "p"
         key -- the right mouse button deselects the greyed pattern.
         Too tricky, we might try fixing it later.
   \end{itemize}

\subsubsection{Performance Editor Time Section}
\label{subsubsec:kbd_mouse_performance_editor_time_section}

   \begin{itemize}
      \item \texttt{l}.  Set to move L marker.
      \item \texttt{r}.  Set to move R marker.
      \item \texttt{x}.  Escape ("x-scape") the movement mode.
      \item \texttt{Left}.  Move the selected marker left.
      \item \texttt{Right}.  Move the selected marker right.
   \end{itemize}

   This section of the performance editor is also known as the "measure ruler"
   or the "bar indicator", and is discussed in
   \sectionref{subsubsec:seq64_song_editor_arrangement_panel_measures_ruler}.
   See \tableref{table:performance_editor_time_section}.

   \begin{table}[H]
      \centering
      \caption{Performance Editor Time Section}
      \label{table:performance_editor_time_section}
      \begin{tabular}{l l l l l l}
         \textbf{Action}   & \textbf{Normal} & \textbf{Double}    & \textbf{Shift} & \textbf{Ctrl}   & \textbf{Mod4}      \\
         l                 & Move L [1]      & ---                & ---            & ---             & ---                \\
         r                 & Move R [1]      & ---                & ---            & ---             & ---                \\
         x                 & Escape Move     & ---                & ---            & ---             & ---                \\
         Left-Click        & Set L [2]       & ---                & ---            & ---             & ---                \\
         Middle-Click      & ---             & ---                & ---            & ---             & ---                \\
         Right-Click       & Set R [2]       & ---                & ---            & ---             & ---                \\
      \end{tabular}
   \end{table}

   \begin{enumerate}
      \item Activates movement of this marker using the left and right arrow
         keys.  Movement is in increments of the snap value.  This mode is
         exited by pressing the 'x' key.  Also see note [2].
      \item Controlled in the pertime section.
   \end{enumerate}

   The new features of this window for \textsl{Sequencer64},
   as noted in
   \sectionref{subsubsec:seq64_song_editor_arrangement_panel_measures_ruler},
   are:

   \begin{itemize}
      \item \textsl{l}:  Enters a mode where the left and right arrow keys move
         the L marker, until the "x" key is pressed.
      \item \textsl{r}:  Enters a mode where the left and right arrow keys move
         the R marker, until the "x" key is pressed.
      \item \textsl{x}:  Exits the marker-movement  mode.
   \end{itemize}

\subsubsection{Performance Editor Names Section}
\label{subsubsec:kbd_mouse_performance_editor_names_section}

   \begin{table}[H]
      \centering
      \caption{Performance Editor Names Section}
      \label{table:performance_editor_names}
      \begin{tabular}{l l l l l l}
         \textbf{Action}   & \textbf{Normal}    & \textbf{Double}    & \textbf{Shift}        & \textbf{Ctrl}   & \textbf{Mod4}      \\
         Left-Click        & Toggle track       & ---                & Toggle other tracks   & ---             & ---                \\
         Middle-Click      & ---                & ---                & ---                   & ---             & ---                \\
         Right-Click       & New/Edit menu      & ---                & ---                   & ---             & ---                \\
      \end{tabular}
   \end{table}

\subsection{Pattern Editor}
\label{subsec:kbd_mouse_pattern_editor}

   The pattern/sequencer editor piano roll is a complex and powerful event
   editor;
   \tableref{table:pattern_editor_piano_roll},
   doesn't begin to cover its functionality.
   Here are the keystrokes handle by the main frame of the window:

   \begin{itemize}
      \item \texttt{Ctrl-W}.  Exit the sequence (pattern) editor.
      \item \texttt{Ctrl-Page Up}.  Zoom in.
      \item \texttt{Ctrl-Page Down}.  Zoom out.
      \item \texttt{Shift-Page Up}.  Scroll leftward.
      \item \texttt{Shift-Page Down}.  Scroll rightward.
      \item \texttt{Shift-Home}.  Scroll leftward to the beginning.
      \item \texttt{Shift-End}.  Scroll rightward to the end.
      \item \texttt{Shift-z (Z)}.  Zoom in.
      \item \texttt{0}.  Set default zoom.
      \item \texttt{z}.  Zoom out.
      \item \texttt{Page Down}.  Scroll downward.
      \item \texttt{Page Up}.  Scroll upward.
      \item \texttt{Home}.  Scroll upward to the beginning.
      \item \texttt{End}.  Scroll downward to the end.
   \end{itemize}

   Where is Delete?

\subsubsection{Pattern Editor Piano Roll}
\label{subsubsec:kbd_mouse_pattern_editor_piano_roll}

   Here are the keystrokes handled by the piano roll:
   These keystrokes require that the focus be set to the piano roll by clicking
   in it with the mouse.

   \begin{itemize}
      \item \texttt{Ctrl-x}. Cut.
      \item \texttt{Ctrl-c}. Copy.
      \item \texttt{Ctrl-v}. Paste.
      \item \texttt{Ctrl-z}. Undo.
      \item \texttt{Ctrl-r}. Redo.
      \item \texttt{Ctrl-a}. Select all.
      \item \texttt{Ctrl-Left}.  Shrink selected notes.
      \item \texttt{Ctrl-Right}.  Grow selected notes.
      \item \texttt{Delete}.  Remove selected notes.
      \item \texttt{Backspade}.  Remove selected notes.
      \item \texttt{Home.  Set sequence to beginnging of sequence}.  (Verify!)
      \item \texttt{Left}.  Move selected notes one snap left.
      \item \texttt{Down}.  Move selected notes one pitch downward.
      \item \texttt{Up}.  Move selected notes one pitch upward.
      \item \texttt{Enter, Return}.
         Paste the selected notes at the current position.
      \item \texttt{p}.  Enter "paint" (also known as "adding") mode.
      \item \texttt{x}.  Escape ("x-scape") the paint mode.
   \end{itemize}

   And here is the table:

   \begin{table}[H]
      \centering
      \caption{Pattern Editor Piano Roll}
      \label{table:pattern_editor_piano_roll}
      \begin{tabular}{l l l l l l}
         \textbf{Action}   & \textbf{Normal} & \textbf{Double}    & \textbf{Shift} & \textbf{Ctrl}   & \textbf{Mod4}      \\
         Del               & Delete Selected & ---                & ---            & ---             & ---                \\
         c                 & ---             & ---                & ---            & Copy            & ---                \\
         p                 & Paint mode      & ---                & ---            & ---             & ---                \\
         v                 & ---             & ---                & ---            & Paste           & ---                \\
         x                 & Escape Paint    & ---                & ---            & Cut             & ---                \\
         z                 & Zoom Out        & ---                & ---            & Undo            & ---                \\
         0                 & Reset Zoom      & ---                & ---            & ---             & ---                \\
         Z                 & Zoom In         & ---                & ---            & Undo?           & ---                \\
         Left-Arrow        & Move Earlier [1] & ---               & ---            & ---             & ---                \\
         Right-Arrow       & Move Later [1]  & ---                & ---            & ---             & ---                \\
         Up-Arrow          & Increase Pitch  & ---                & ---            & ---             & ---                \\
         Down-Arrow        & Decrease Pitch  & ---                & ---            & ---             & ---                \\
         Left-Click        & Deselect        & ---                & ---            & ---             & ---                \\
         Right-Click       & Paint mode      & ---                & Edit Menu      & Edit/Edit Menu  & Lock Paint mode    \\
         Left-Middle-Click & Grow Selected   & ---                & Stretch Sel.   & ---             & ---                \\
         Scroll-Up         & Zoom Time In    & ---                & Scroll Left    & Zoom Time In    & ---                \\
         Scroll-Down       & Zoom Time Out   & ---                & Scroll Right   & Zoom Time Out   & ---                \\
      \end{tabular}
   \end{table}

   \begin{itemize}
      \item Once selected (and thus rendered in grey), a pattern segment
         can be moved by the mouse.  To move it using the left or right
         arrow keys, the paint mode must be entered, but only via the
         \texttt{p} key -- the right mouse button deselects the greyed pattern.
         Too tricky, we might try fixing it later.
   \end{itemize}

   The new features of this window section for \textsl{Sequencer64},
   as noted in
   \sectionref{subsubsec:seq64_pattern_editor_piano_roll_items}, are:

   \begin{itemize}
      \item \textsl{p}:  Enters the paint mode, until right-click is pressed or
         until the \texttt{x} key is pressed.
      \item \textsl{x}:  Exits the paint mode.  Think of the made-up term
         "x-scape".
      \item \textsl{z}:  Zooms out the performance view.  It makes the view
         look smaller, so that more of the performance can be seen.  Please
         note that opening a second performance view is another way to see more
         of the performance.
      \item \textsl{0}:  Resets the performance view zoom to its normal value.
      \item \textsl{Z}:  Zooms in the performance view.  It makes the view
         look bigger, so that more details of the performance can be seen.
      \item \textsl{.}:  The period (configurable) is a new key devoted to the
         new pause functionality.
      \item \textsl{Left Arrow}:  Moves the selected events to the left (earlier
         in time) in the performance layout.
      \item \textsl{Right Arrow}:  Moves the selected events to the right (later
         in time) in the performance layout.
      \item \textsl{Up Arrow}:  Moves the selected notes upward in direction
         and pitch.
      \item \textsl{Down Arrow}:  Moves the selected notes downward in
         direction and pitch.
      \item \textsl{Mod4-Right-Click}:  Locks the paint mode, until right-click
         is pressed again later.
   \end{itemize}

\subsubsection{Pattern Editor Event Panel}
\label{subsubsec:kbd_mouse_pattern_editor_event_panel}

   \begin{itemize}
      \item \texttt{Ctrl-x}. Cut.
      \item \texttt{Ctrl-c}. Copy.
      \item \texttt{Ctrl-v}. Paste.
      \item \texttt{Ctrl-z}. Undo.
      \item \texttt{Delete}.  Delete (not cut!) the selected events.
      \item \texttt{p}.  Enter "paint" (also known as "adding") mode.
      \item \texttt{x}.  Escape ("x-scape") the paint mode.
   \end{itemize}

\subsubsection{Pattern Editor Data Panel}
\label{subsubsec:kbd_mouse_pattern_editor_data_panel}

   TO BE DETERMINED.

\subsubsection{Pattern Editor Virtual Keyboard}
\label{subsubsec:kbd_mouse_pattern_editor_virtual_keyboard}

   \begin{table}[H]
      \centering
      \caption{Pattern Editor Virtual Keyboard}
      \label{table:pattern_editor_virtual_keyboard}
      \begin{tabular}{l l l l l l}
         \textbf{Action}   & \textbf{Normal} & \textbf{Double}    & \textbf{Shift} & \textbf{Ctrl}   & \textbf{Mod4}      \\
         Left-Click        & Play note       & ---                & ---            & ---             & ---                \\
         Right-Click       & Toggle labels   & ---                & ---            & ---             & ---                \\
      \end{tabular}
   \end{table}

\subsection{Event Editor}
\label{subsec:kbd_mouse_event_editor}

   \begin{itemize}
      \item \texttt{Down}.  Move one slot down.
      \item \texttt{Up}.  Move one slot up.
      \item \texttt{Page Down}.  Move one frame down.
      \item \texttt{Page Up}.  Move one frame up.
      \item \texttt{Home}.  Move to top frame.
      \item \texttt{End}.  Move to bottom frame.
      \item \texttt{Asterisk, KP Multiply}.  Delete the currently-selected event.
   \end{itemize}

%-------------------------------------------------------------------------------
% vim: ts=3 sw=3 et ft=tex
%-------------------------------------------------------------------------------
