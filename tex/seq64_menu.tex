%-------------------------------------------------------------------------------
% seq64_menu
%-------------------------------------------------------------------------------
%
% \file        seq64_menu.tex
% \library     Documents
% \author      Chris Ahlstrom
% \date        2015-08-31
% \update      2016-01-09
% \version     $Revision$
% \license     $XPC_GPL_LICENSE$
%
%     Provides the Menu section of seq24-user-manual.tex.
%
%-------------------------------------------------------------------------------

\section{Menu}
\label{sec:seq64_menu}

   The \textsl{Sequencer64} menu, as seen at the top of
   \figureref{fig:seq64_main_screen},
   is fairly simple, but it is important to understand the
   structure of the menu entries.

\subsection{Menu / File}
\label{subsec:seq64_menu_file}

   The \textbf{File} menu is used to save and load standard MIDI files.
   \textsl{Sequencer64}
   should be able to handle any Format 1 standard files that any other
   sequencer is capable of exporting.  

   The \textsl{Sequencer64} menu entry contains the sub-items shown in
   \figureref{fig:seq64_menu_file_items}.  The next few sub-sections discuss
   the sub-items in the \textsl{File} sub-menu.

\begin{figure}[H]
   \centering 
   \includegraphics[scale=0.75]{menu/menu_file.png}
   \caption{Sequencer64 File Menu Items}
   \label{fig:seq64_menu_file_items}
\end{figure}

   \begin{enumber}
      \item \textbf{New}
      \item \textbf{Open...}
      \item \textbf{Save}
      \item \textbf{Save As...}
      \item \textbf{Import...}
      \item \textbf{Options...}
      \item \textbf{Exit}
   \end{enumber}

\subsection{Menu / File / New}
\label{subsec:menu_file_new}

   The \textbf{New} menu entry clears out any current song and patterns,
   allowing one to create news ones from scratch.
   If unsaved changes are pending, the user will be prompted to save the
   changes.

   \index{todo!improve change detection}
   \textsl{Currently, the detection of situations requiring a save (or not
   requiring a save) needs a bit of work.}

\subsubsection{Menu / File / Open}
\label{subsubsec:seq64_menu_file_open}

   The \textbf{Open} menu entry opens a song (MIDI file)
   that had been saved previously.  It opens up a standard GTK+2 file dialog:

\begin{figure}[H]
   \centering 
   \includegraphics[scale=0.50]{menu/menu_file_open.png}
   \caption{File Open}
   \label{fig:seq64_menu_file_open}
\end{figure}

   If unsaved changes are pending, the user will (usually)
   be prompted to save the changes.  When in doubt, save!
   If still in doubt, keep backups of your tunes!

\subsubsection{Menu / File / Save and Save As}
\label{subsubsec:menu_file_open_save_as}

   The \textbf{Save} menu entry saves the song under its current file-name.
   If there is no current file-name, then it opens up a standard GTK+2 file
   dialog to name and save the file.

   The \textbf{Save As} menu entry saves a song under a different name.
   It opens up the following standard GTK+2 file dialog:

\begin{figure}[H]
   \centering 
   \includegraphics[scale=0.50]{menu/menu_file_save_as.png}
   \caption{File Save As}
   \label{fig:seq64_menu_file_save_as}
\end{figure}

   To save a new file, or to save the current existing file to a new name,
   enter the name in the name field, \textsl{without an extension}.
   \textsl{Sequencer64} will append a \texttt{.midi} extension to the filename.
   The file will be saved in a format that the Linux \textsl{file} command
   will tag as something like:

   \begin{verbatim}
      myfile.midi: Standard MIDI data (format 1) using 16 tracks at 1/192
   \end{verbatim}

   It looks like a simple MIDI file, and yet, if one re-opens it in
   \textsl{Sequencer64}, one sees that all of the labelling, pattern information,
   and song layout has been preserved in this file.
   Even the pattern subsections, as discussed in
   \sectionref{subsubsec:seq64_song_editor_arrangement_panel_roll},
   have been saved.
   (But the L and R marker positions are not saved.)

   Compare the sizes of the original project MIDI file,
   \texttt{contrib/b4uacuse.mid}, and the output MIDI file after
   \textsl{Sequencer64} saved the patterns and the song layout we created,
   \texttt{contrib/b4uacuse-seq24.midi}.  The latter is a lot
   bigger.

   The reason is that, after the last track in the file, a number of
   sequencer-specific (SeqSpec) items are saved, to preserve this extra
   information.  In legacy mode, \textsl{Sequencer64} saves this information
   in the same format as \textsl{Seq24}.  Unfortunately, this format is
   not quite standard, and a few MIDI applications may produce error
   messages (as opposed to just ignoring it) when parsing this section. 
   
   \textbf{New:}
   \index{new!seqspec format}
   Therefore, in its normal mode, \textsl{Sequencer64} saves this
   information in a more MIDI-compliant format, marking each SeqSpec section
   as vendor-specific information, and marking this section as a regular
   MIDI track.
   The legacy and new formats of the final "track" are explained in
   \sectionref{subsec:legacy_midi_format}.

\subsubsection{Menu / File / Import}
\label{subsubsec:seq64_menu_file_import}

   The \textbf{Import} menu entry allows one to import a Format 1 MIDI file
   into one or more patterns, one pattern per track in the MIDI file.
   Even long tracks, that aren't short loops, are read in properly.

\begin{figure}[H]
   \centering 
   \includegraphics[scale=0.50]{menu/menu_file_import.png}
   \caption{File Import}
   \label{fig:seq64_menu_file_import}
\end{figure}

   When imported, each track, whether a music track or an information track,
   is entered into its own loop/pattern box.  The import operation can
   handle reasonably complex files, as shown in the following diagram, which
   shows an import of the \texttt{contrib/b4uacuse.mid} file, which contains
   a transcription of an Eric Clapton tune that we'd made over 20 
   years ago and had uploaded to the \textsl{GEnie} network service.

   Note the additional file-dialog field,
   \textbf{Select Screen Offset}.
   \index{import!select screen offset}
   \index{select screen offset}
   This setting lets one place the imported data into a screen-set other than
   the first screen-set, screen-set 0.
   This field is not editable.  It requires using the scroll button to move the
   screen set offset up or down in value.  The legal values range from -31 to 0
   to +31.
   
   When the file is imported, the sequence number for each track read in is
   adjusted to put the track in the desired screen set.  The negative numbers
   are probably more useful to move sequences around in an already-created
   \textsl{Sequencer64} song file with a lot of screen-sets in it.

\begin{figure}[H]
   \centering 
   \includegraphics[scale=0.90]{menu/imported_midi_song.png}
   \caption{Imported MIDI Song}
   \label{fig:seq64_imported_midi_song}
\end{figure}

   This song has a number of non-event tracks containing labels.
   \textbf{New:}
   \index{new!empty tracks}
   These tracks are treated as "empty", and show up with a yellow
   background.  They are ignored on playback.   Some of the tracks
   which look empty, but are not yellow, contain program-change events.

   Unfortunately, this song was created before the days of General MIDI.
   It was scored for the Yamaha PSS-790 consumer-level synthesizer,
   definitely not a GM-compliant device.
   One can use the MIDI-conversion project, \textsl{midicvt}
   (see reference \cite{midicvt}),
   to convert it to General MIDI format, including General MIDI drums.

\subsubsection{Menu / File / Options}
\label{subsubsec:seq64_menu_file_options}

   The \textbf{Options} menu item provides a number of settings in one
   tabbed dialog, shown in the figures that follow.
   This dialog allows one to select which sequence gets the MIDI
   clock, which incoming MIDI events control the sequencer, what keys are
   mapped to functions, how the mouse works, and some JACK parameters.

\paragraph{Menu / File / Options / MIDI Clock}
\label{paragraph:seq64_menu_file_options_midi_clock}

   The \textbf{MIDI Clock} tab provides a way to send the MIDI clock to one
   or more of the \textsl{Sequencer64} output busses.
   It is used to configure to what busses the MIDI clock gets dumped.
   It also shows the devices that one can play music with.
   The items that appear in this tab depend on three things:

   \begin{itemize}
      \item What MIDI devices are connected to the computer.  For example,
         MIDI controllers, USB MIDI cables, and other devices will add MIDI
         output devices (ports) to the system.
      \item What MIDI software devices are running on the computer.
         For example, running MIDI software synthesizers such as
         \textsl{Timidity} and \textsl{Yoshimi} will add extra output devices
         (playback ports) to a system.
      \item The setting of the "manual ALSA ports" option,
         \texttt{--manual\_alsa\_ports} command-line option or the
         \texttt{[manual-alsa-ports]} section of the
         \texttt{sequencer64.rc} configuration file, as described in
         \sectionref{subsec:seq64_rc_file_other_midi}
   \end{itemize}

   For the current discussion, a USB MIDI cable was plugged into the system,
   and the \textsl{Timidity} and \textsl{Yoshimi} (in ALSA mode) software
   synthesizers were running.  \textsl{Sequencer64} was also running, of
   course.  Here are the devices shown by the ALSA MIDI playback
   command-line application:

   \begin{verbatim}
      $ aplaymidi -l
       Port    Client name                      Port name
       14:0    Midi Through                     Midi Through Port-0
       24:0    USB2.0-MIDI                      USB2.0-MIDI MIDI 1
       24:1    USB2.0-MIDI                      USB2.0-MIDI MIDI 2
      128:0    TiMidity                         TiMidity port 0
      128:1    TiMidity                         TiMidity port 1
      128:2    TiMidity                         TiMidity port 2
      128:3    TiMidity                         TiMidity port 3
      130:16   seq24                            seq24 in
   \end{verbatim}

   (For some reason, the \textsl{Yoshimi} input port is not showing up
   in the output of \texttt{aplaymidi}, though, as shown in
   \figureref{fig:seq64_midi_clock_4_devices_manual_0},
   \textsl{Sequencer64} sees it on port 7.  Perhaps that application is not
   providing a good ALSA device name.)
   
   Turning to \figureref{fig:seq64_midi_clock_4_devices_manual_1},
   note the 16 devices provided by
   \textsl{Sequencer64}.  Also note that its first value is 1, not 0, due to
   the MIDI Thru port occupying slot 0.
   This figure shows the result with the manual ALSA option of
   \textsl{Sequencer64} turned on.

\begin{figure}[H]
   \centering 
   \includegraphics[scale=0.75]{menu/midi-clock-4-devices-manual-1.png}
   \caption{MIDI Clock, Manual ALSA Option On}
   \label{fig:seq64_midi_clock_4_devices_manual_1}
\end{figure}

   It basically shows the 16 MIDI output busses that \textsl{Sequencer64} can
   drive.  One would have to use an ALSA MIDI connection application to put a
   device on each of those outputs.  The fact that the the buss names can
   start with different numbers, depending on the system setup, can complicate
   the playing of MIDI in this manner.

   The following elements are present in this dialog:

   \begin{enumber}
      \item \textbf{Buss Name}
      \item \textbf{Off}
      \item \textbf{On (Pos)}
      \item \textbf{On (Mod)}
      \item \textbf{Clock Start Modulo}
   \end{enumber}

   \setcounter{ItemCounter}{0}      % Reset the ItemCounter for this list.

   \itempar{Buss Name}{midi clock!buss name}
   \index{port name}
   \index{midi clock!port name}
   These labels indicate the output busses (ports) of \textsl{Sequencer64}.
   They range from \textbf{[1] seq24 1} to \textbf{[16] seq24 16}.

   \itempar{Off}{midi clock!off}
   This setting disables the MIDI clock for the given output buss.
   However, note that MIDI output can still be sent to those ports, and
   each port that has a device connected to it will play music.
   
   For feeding \textsl{Yoshimi} with MIDI data, we found that this
   setting is the one that must be made in order for \textsl{Yoshimi} to
   produce a sound.

   \itempar{On (Pos)}{midi clock!on (pos)}
   The MIDI clock will be sent to this buss.
   MIDI Song Position and MIDI Continue will be sent if playback is starting
   at greater than tick 0 in Song mode.  Otherwise, MIDI Start will be sent.

   \itempar{On (Mod)}{midi clock!on (mod)}
   The MIDI clock will be sent to this buss.
   MIDI Start will be sent and clocking will begin
   once the Song Position has reached the start modulo of the specified size
   (see the next item's description).
   This setting is used for gear that does not respond to Song Position.

   \itempar{Clock Start Modulo}{midi clock!clock start modulo}
   Clock Start Modulo (1/16 Notes).
   This value starts at 1 and ranges on upward to 16384.
   It  defaults to 64.
   It is used by the \textbf{On (Mod)} setting discussed above.
   It is the \texttt{[midi-clock-mod-ticks]} option in the \textsl{Sequencer64}
   "rc" file as described in
   \sectionref{subsec:seq64_rc_file_other_midi}.

\begin{figure}[H]
   \centering 
   \includegraphics[scale=0.75]{menu/midi-clock-4-devices-manual-0.png}
   \caption{MIDI Clock, Manual ALSA Option Off}
   \label{fig:seq64_midi_clock_4_devices_manual_0}
\end{figure}

   As shown by the figure above, with the manual ALSA option turned off,
   all of the devices that can be driven by MIDI output are shown,
   including the MIDI Thru port, the two MIDI ports on the USB cable,
   the four ports provided by \textsl{Timidity}, and the unlabelled
   port provided by \textsl{Yoshimi}.

   One could theoretically play music through 6 or 7 devices using
   \textsl{Sequencer64} with this setup.

   \textbf{TODO:} \index{todo!manual alsa gui option}
   There is currently no user-interface item corresponding to this command-line
   and "rc" configuration file option.

\paragraph{Menu / File / Options / MIDI Input}
\label{paragraph:seq64_menu_file_options_midi_input}

   To allow \textsl{Sequencer64} to record MIDI from MIDI devices such as
   controllers and keyboards, the output of the ALSA MIDI recording
   command-line application is relevant:

   \begin{verbatim}
      $ arecordmidi -l
       Port    Client name                      Port name
       14:0    Midi Through                     Midi Through Port-0
       24:0    USB2.0-MIDI                      USB2.0-MIDI MIDI 1
      130:0    seq24                            [1] seq24 1
      130:1    seq24                            [2] seq24 2
      130:2    seq24                            [3] seq24 3
       . . .   . . .                               . . .
      130:15   seq24                            [16] seq24 16
   \end{verbatim}

   We see that we can record MIDI from the MIDI Thru port, from the USB MIDI
   cable, and MIDI from any of the 16 output ports provided by the manual ALSA
   port mode of \textsl{Sequencer64}.

   If the "manual ALSA ports" option (see below) is turned on,
   then the only item in the \textbf{MIDI Input} tab is the single MIDI input
   buss provided by \textsl{Sequencer64}:  \textbf{[0] seq24 0}, or, since
   the MIDI Thru port takes slot 0, \textbf{[1] seq24 1}.

\begin{figure}[H]
   \centering 
   \includegraphics[scale=0.75]{menu/midi-input-4-devices-manual-1.png}
   \caption{MIDI Input, Manual ALSA Ports On}
   \label{fig:seq64_midi_input_4_devices_manual_1}
\end{figure}

   This item, if checked, allows \textsl{Sequencer64} to be used to record MIDI
   information from another source (which must be connected to this port by
   another application), or pass it through to the output busses
   that are configured to allow pass-through
   (in the Pattern Editor, as discussed in 
   \sectionref{subsec:seq64_pattern_editor_bottom}.)

   If the "manual ALSA ports" option is turned off, then
   the input ports from the system are shown:

\begin{figure}[H]
   \centering 
   \includegraphics[scale=0.75]{menu/midi-input-4-devices-manual-0.png}
   \caption{MIDI Input, Manual ALSA Ports Off}
   \label{fig:seq64_midi_input_4_devices_manual_0}
\end{figure}

   For example, one could check input \#1 to have \textsl{Sequencer64} record
   MIDI from an old-fashioned MIDI keyboard that is connected to the USB MIDI
   cable.  If the keyboard didn't have a sound generator, one would also want
   \textsl{Sequencer64} to pass this MIDI on to a sound generator, such as a
   software or hardware synthesizer attached to one of the ports shown in
   \figureref{fig:seq64_midi_clock_4_devices_manual_0}.

\paragraph{Menu / File / Options / Keyboard }
\label{paragraph:seq64_menu_file_options_keyboard}

   \textsl{Sequencer64}, as befits a good application, allows extensive use of
   keyboard shortcuts to make operations go faster than when using a mouse.
   The \textbf{Keyboard} tab allows for the configuration of these keyboard
   shortcuts.

   \textbf{Warning:}
   Whenever one of the text fields in this dialog has the focus (and that is
   usually the case), then any keystroke, including keys like Ctrl, Alt,
   and Super (Mod4 or Windows key), can alter the value of a field to that
   of the keystroke.  This change is very easy to do accidentally!
   \textbf{Use the mouse}
   to move this window and to click its \textbf{OK} button!

\begin{figure}[H]
   \centering 
   \includegraphics[scale=0.75]{menu/menu_file_options_keyboard.png}
   \caption{File / Options / Keyboard}
   \label{fig:seq64_menu_file_options_keyboard}
\end{figure}

   We won't attempt to cover every user-interface item in this busy
   dialog, just the categories.  Some items are discussed in other parts of
   this manual.

   \begin{enumber}
      \item \textbf{Show key labels on sequences}
      \item \textbf{Show sequence numbers on sequences}
      \item \textbf{Control keys}
      \item \textbf{Sequence toggle keys}
      \item \textbf{Mute-group slots}
      \item \textbf{Learn}
      \item \textbf{Disable}
      \item \textbf{Enable}
   \end{enumber}

   \setcounter{ItemCounter}{0}      % Reset the ItemCounter for this list.

   \itempar{Show key labels on sequence}{keyboard!show labels}
   This item, if enabled, shows the key labels in the lower-right corner of
   each loop/pattern in the Patterns window (the main window).  This feature is
   useful for live playback and control of a song.
   Note that this option is also available in the "rc" configuration file.

   \itempar{Show sequence numbers on sequence}{keyboard!sequence numbers}
   \textbf{New:}
   \index{new!sequence numbers}
   If this option is turned on, the
   empty slots in the Patterns window show the prospective sequence number.
   See the following figure for the look.

\begin{figure}[H]
   \centering 
   \includegraphics[scale=0.75]{pattern-window-with-numbering.png}
   \caption{Pattern Window with Numbering}
   \label{fig:seq64_build_with_numbering}
\end{figure}

   If you don't like it, turn off the option, or try other grid options
   in the "user" configuration file.

   Also note that this option also changes the visibility of sequence numbers
   in active sequences and in the performance editor's names column.

   \itempar{Control keys}{keyboard!control keys}
   This block of fields provides shortcut keys for many operations of
   \textsl{Sequencer64}.

   \begin{enumber}
      \item \textbf{Start}.
         Key: \index{keys!space} \textbf{space}.
      \item \textbf{Stop}.
         Key: \index{keys!esc} \textbf{Escape}.
      \item \textbf{Snapshot 1}.
         Key: \index{keys!alt-l} \textbf{Alt\_L}.
      \item \textbf{Snapshot 2}.
         Key: \index{keys!alt-r} \textbf{Alt\_R}.
      \item \textbf{bpm up}.
         Key: \index{keys!apostrophe} \textbf{apostrophe}.
      \item \textbf{bpm down}.
         Key: \index{keys!semicolon} \textbf{semicolon}.
      \item \textbf{Replace}.
         Key: \index{keys!ctrl-l} \textbf{Control\_L}.
      \item \textbf{Queue}.
         Key: \index{keys!ctrl-r} \textbf{Control\_R}.
      \item \textbf{Keep queue}.
         Key: \index{keys!backslash} \textbf{backslash}.
      \item \textbf{Screenset down}.
         Key: \index{keys![} \textbf{bracketleft}.
      \item \textbf{Screenset up}.
         Key: \index{keys!]} \textbf{bracketright}.
      \item \textbf{Set playing screenset}.
         Key: \index{keys!home} \textbf{Home}.
   \end{enumber}

   Note that some of the keys have positional mnemonic value.  For example,
   for BPM control, the semicolon is at the left (down), and the apostrophe
   is at the right (up).

   Also note that the keys definable in this tab are only a subset of the
   various keys that can be used, especially keys used with the
   \texttt{Ctrl} key or other modifier keys.

   \index{snapshot}
   A \textsl{snapshot} is a briefly preserved state of the patterns.
   One can press a snapshot key, change the state of the patterns for live
   playback, and then release the snapshot key to revert to the state when
   the snapshot key was first pressed.

   \index{queue}
   To "queue" a pattern means to ready it for playback upon the next repeat
   of a pattern.  A pattern can be armed immediately, or it can be queued to
   play back the next time the pattern starts.

   \index{keep queue}
   \index{queue!keep}
   The "keep queue" functionality allows the queue to be held without
   holding down a button the whole time.

   \itempar{Sequence toggle keys}{keyboard!sequence toggle keys}
   Each of these keys toggles the playing/muting of one of the 32
   loop/pattern boxes.  These keys are layed out logically on the keyboard,
   and can also be shown in each loop/pattern box.  No need to list them all
   here!

   \itempar{Mute-group slots}{keyboard!mute-group slots}
   Each of these keys operates on the mute-grouping of one of the 32
   loop/pattern boxes.  These keys are layed out logically on the keyboard,
   and can also be shown in each loop/pattern box.  No need to list them all
   here!

   Apparently groups work with the playing screen set only.
   Change the screenset and give the command to make it the playing one
   (e.g. set the HOME key for this purpose.)

   \itempar{Learn}{keyboard!learn}
   Learn (while pressing a mute-group key).
   This items sets the key used to initiate a learn mode.
   It is the \textbf{Insert} key by default.

   \itempar{Disable}{keyboard!disable}
   \textbf{TODO:} \index{todo!keyboard disable} What gets disabled?
   \index{keys!apostrophe}
   It is the \textbf{apostrophe} key by default.

   \itempar{Enable}{keyboard!enable}
   \textbf{TODO:} What gets enabled?
   \index{keys!igrave}
   It is the \textbf{igrave} (back-tick) key by default.

   There is much to learn about this learn/enable/disable triad!

\paragraph{Menu / File / Options / Mouse }
\label{paragraph:seq64_menu_file_options_mouse}

   This item selects the mouse-interaction method.

\begin{figure}[H]
   \centering 
   \includegraphics[scale=0.75]{menu/menu_file_options_mouse_condensed.png}
   \caption{File / Options / Mouse (Condensed View)}
   \label{fig:seq64_menu_file_options_mouse}
\end{figure}

   The default method is \textbf{seq24 (original style)}.
   The alternate method is \textbf{fruity (similar to a certain well known
   sequencer)}.

   The alternate method is presumably that of the \textsl{Fruity Loops}
   (now \textsl{FL Studio}) sequencer.  The fruity mode seems to involve the
   following (based on scanning the source code):
   
   \begin{itemize}
      \item \textbf{Left-click left side}.
         Begin a grow/shrink operation for the left side.
      \item \textbf{Left-click right side}.
         Begin a grow/shrink operation for the right side.
      \item \textbf{Left-click middle}.
         Move the object.
      \item \textbf{Left-click}.
         Add an event if nothing selected.
      \item \textbf{Middle-click}.
         Split the note?
   \end{itemize}

   The \textsl{Seq24} "original style" is pretty much as expected for basic
   actions such as selecting and moving notes using the left mouse button.
   Drawing a note or event is a bit different, in the one must first
   \textsl{click and hold} the right mouse button, and then
   \textsl{click and drag} the right mouse button to insert notes,
   Notes are inserted to be at the current length and grid-snap values for
   the sequence editor for as long as the left button is pressed.
   Notes are inserted only up to the boundary of the sequence length.
   And, once notes are inserted, moving the mouse with the left button still
   held down simply moves the notes to the new note value of the mouse.

   If one releases the left button, then presses and holds it again,
   more notes will be added in the same way.
   This is strange, but it is a powerful way to layer notes into a short
   sequence.
   We call it the \index{draw mode} \index{mode!draw } "draw mode" or
   \index{paint mode} \index{mode!paint } "paint mode".

   Note that drawing/painting can also be done while the sequence is playing,
   and notes will be added to be played the next time the progress bar crosses
   them.
   
   \textbf{New:}
   \index{new!Mod4 mode}
   \index{keys!Mod4}
   \label{new_mod4_mode}
   \textbf{The Mod4 Right-Click Mode}.
   In order to work better with certain trackpads, the
   "Seq24" mode of mouse interaction can be modified (only in the
   Pattern Editor at present) so that the Mod4 key (Super or Windows key)
   can be pressed when releasing the right mouse button.
   This keeps the mouse in note-add mode.
   Another right-click, without pressing Mod4, will exit this mode.

   The reason for this feature is the crummy FocalTech touchpad on one of
   the author's laptops.  This trackpad seems to have only a single button,
   which the driver interprets as left or right depending where the finger
   is when it is clicked.  There's no way to click the right and left
   buttons at the same time.  There's no way to make a middle-click action.

   Note that this option will not interfere with the Mod4 key being set
   in the \textbf{Keyboard} option tab, since the keys there mainly apply to
   the Patterns Panel (main window).

% Move this section to the right place and simply create a section-reference to
% it here.

   \textbf{New:}
   \index{new!paint mode}
   Another way to turn on the paint mode has been added, based on a feature
   found in a patch that someone posted about in some mailing list somewhere on
   the internet.
   To turn on the paint mode, press the
   \index{keys!p}
   "p" key while in the sequence editor.
   This is just like pressing the right mouse button, but the draw/paint mode
   sticks (as if the Mod4 mode were in force).
   To get out of the paint mode, press the
   \index{keys!x}
   "x" key while in the sequence editor.
   These keys, however, do not work (currently) while the sequence is playing.

   \index{todo:extend mouse support}
   \textbf{TODO:} These convenience options are currently limited to the
   pattern/sequence editor window and the performance editor window, and may
   need some heavier testing.  But note that some \textsl{Sequencer64} windows
   can use the ctrl-left-click as a middle click. 
 
\paragraph{Menu / File / Options / Jack Sync and LASH}
\label{paragraph:seq64_menu_file_options_jack_sync}

   This tab sets up options for JACK synchronization, if \textsl{Sequencer64}
   was built with JACK support.  (Why wouldn't it be?)
   This tab also sets up options for using LASH session management, if
   \textsl{Sequencer64} was build with LASH support.

\begin{figure}[H]
   \centering 
   \includegraphics[scale=0.75]{menu/menu_file_options_jack_sync.png}
   \caption{File / Options / JACK Sync or JACK/LASH}
   \label{fig:seq64_menu_file_options_jack_sync}
\end{figure}

   \begin{enumber}
      \item \textbf{Transport}
      \item \textbf{JACK Start mode}
      \item \textbf{Connect}
      \item \textbf{Disconnect}
      \item \textbf{LASH Options}
   \end{enumber}

   \setcounter{ItemCounter}{0}      % Reset the ItemCounter for this list.

   \itempar{Transport}{jack sync!transport}
   These settings are stored in the "rc" file settings group
   \texttt{[jack-transport]}.
   See \sectionref{subsec:seq64_rc_file_jack_transport},
   which describes this configuration option.
   This items collects the following settings:

   \begin{itemize}
      \item \textbf{Jack Transport}.
         \index{JACK!transport}
         Enables synchronization with JACK Transport.
         The command-line option is \texttt{--jack\_transport}.
      \item \textbf{Transport Master}.
         \index{JACK!transport master}
         \textsl{Sequencer64} will attempt to serve as the JACK Master.
         The command-line option is \texttt{--jack\_master}.
      \item \textbf{Master Conditional}.
         \index{JACK!master conditional}
         \textsl{Sequencer64} will fail to serve as the JACK Master if there is
         already a Master set.
         The command-line option is \texttt{--jack\_master\_cond}.
   \end{itemize}

   \itempar{JACK Start mode}{jack sync!start mode}
   This items collects the following settings, also stored in the "rc" file
   settings group \texttt{[jack-transport]}.

   \begin{itemize}
      \item \textbf{Live Mode}.
         \index{JACK!live mode}
         \index{live mode}
         \index{non-playback mode}
         Playback will be in live mode.  Use this option to allow muting and
         unmuting of patterns.  This option might also be called "non-playback
         mode".
         The command-line option is \texttt{--jack\_start\_mode 0}.
      \item \textbf{Song Mode}.
         \index{JACK!song mode}
         \index{song mode}
         \index{playback mode}
         \index{performance mode}
         Playback will use only the Song Editor's data.
         The command-line option is \texttt{--jack\_start\_mode 1}.
   \end{itemize}

   Note that, in ALSA mode, \textsl{Sequencer64} selects the following modes
   according to which window started the playback.  The main window, or pattern
   window, causes playback to be in live mode.  The user can arm and mute
   patterns in that windows, by clicking on sequences, using their hot-keys,
   and by using the group-mode and learn-mode features (I think!).

   The song editor causes playback to be in performance mode, also known as
   playback mode.

   (It remains to be determined if this feature also holds in JACK 
   mode; it shouldn't. One would think that the JACK Start mode
   is then the arbiter of playback.)

   \itempar{Connect}{jack sync!connect}
   Connect to JACK Sync.

   \itempar{Disconnect}{jack sync!disconnect}
   Disconnect from JACK Sync.

   \itempar{LASH Options}{lash!option}
   Currently contains only one item, which enables the usage of LASH session
   management.
   Currently, \textsl{Sequencer64} needs to be restarted to complete the
   enabling or disabling of LASH support.
   Like the rest of the options, this one is written to the "rc" configuration
   file.

\subsection{Menu / View}
\label{subsec:seq64_menu_view}

   If the "allow two perfedits" option is turned off in the "user"
   configuration file, this menu item has only one entry, \textbf{Song Editor}, 
   which is already covered by a button at the bottom of the Patterns
   window.  Selecting this item bring up the Song Editor window.
   See \figureref{fig:song_editor_window}

   The Song Editor window can also be brought up via the
   \index{song editor!ctrl-e}
   \index{keys!ctrl-e}
   Ctrl-E key.

   If the "allow two perfedits" option is turned on in the "user"
   configuration file, this menu item has two entries, as shown in the
   following figure:

\begin{figure}[H]
   \centering 
   \includegraphics[scale=0.75]{menu/menu_view-dual-song-editors.png}
   \caption{Dual Song Editor Entries in View Menu}
   \label{fig:seq64_menu_view_song_editors}
\end{figure}

   Note that only the first Song Editor has a user-interface button and
   a hot-key.  Also note that there can be issues bringing up the second
   song-editor with the hot-key.  The menu entry will always work.

   If two song editors are up, they each track any changes made in the other
   song editor.  But the main purpose of two song editors is to arrange two
   different parts of the performance at the same time.

\subsection{Menu / Help About...}
\label{subsec:seq64_menu_about}

   This menu entry shows the "About" dialog.

\begin{figure}[H]
   \centering 
   \includegraphics[scale=0.75]{menu/menu_help_about.png}
   \caption{Help About}
   \label{fig:seq64_menu_help_about}
\end{figure}

   That dialog provides access to the credits for the program, including the
   authors and the project documentors.

\begin{figure}[H]
   \centering 
   \includegraphics[scale=0.75]{menu/menu_help_credits.png}
   \caption{Help Credits}
   \label{fig:seq64_menu_help_credits}
\end{figure}

   Shows who has worked on the program, with the original author at the top
   of the list.

\begin{figure}[H]
   \centering 
   \includegraphics[scale=0.75]{menu/menu_help_doc.png}
   \caption{Help Documentation}
   \label{fig:seq64_menu_help_doc}
\end{figure}

   Shows who has documented this project.

%-------------------------------------------------------------------------------
% vim: ts=3 sw=3 et ft=tex
%-------------------------------------------------------------------------------
