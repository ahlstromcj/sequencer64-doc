%-------------------------------------------------------------------------------
% seq64_event_editor
%-------------------------------------------------------------------------------
%
% \file        seq64_event_editor.tex
% \library     Documents
% \author      Chris Ahlstrom
% \date        2016-01-02
% \update      2016-01-02
% \version     $Revision$
% \license     $XPC_GPL_LICENSE$
%
%-------------------------------------------------------------------------------

\section{Event Editor}
\label{sec:seq64_event_editor}

   The \textsl{Sequencer64 Event Editor} is used to view and edit,
   in detail, the events present in a sequence/pattern/track.

   \textbf{Warning:}
   \index{warning!event editor}
   This dialog is a new feature of \textsl{Sequencer64}, and is
   still a work-in-progress.  Basic viewing and scrolling generally work well,
   and editing, deleting, and inserting events does work.
   But this dialog has not yet had rigorous testing, and it surely
   has some nasty bugs lurking in it.
   If anything bad happens, do \textsl{not} press the
   \textbf{Save to Sequence} button!

   Also note that this editor is not very sophisticated.  It requires the user
   to know about MIDI events and data values.  It does not present handy
   dropdown lists for various items.
   It does not detect any changes made to the sequence in the pattern editor.
   If, some day, we find ourselves missing
   that kind of functionality, then we can add it.

   For now, the event editor is a good way to see the events in a sequence.

\begin{figure}[H]
   \centering 
   \includegraphics[scale=0.75]{event-editor/preliminary-event-editor.png}
   \caption{Event Editor Window}
   \label{fig:event_editor_window}
\end{figure}

   This dialog is fairly complex.
   For exposition, we break it into a few parts:

   \begin{enumber}
      \item \textbf{Event Frame}
      \item \textbf{Info Panel}
      \item \textbf{Edit Fields}
      \item \textbf{Bottom Buttons}
   \end{enumber}

   The event frame consists of a list of events, which can be
   viewed, traversed, and edited.  The info fields show the name of the
   sequence containing the events, and some other information about the
   sequence.  The edit fields provide four text fields for viewing and entering
   information about an event, and buttons to delete, insert, and modify
   events.  The bottom buttons allow changes to be saved and the editor to be
   closed.  

   The following sections described these items in detail.

\subsection{Event Editor / Event Frame}
\label{subsec:seq64_event_editor_frame}

\subsubsection{Event Frame / Data Items}
\label{subsec:seq64_event_frame_data}

   The event frame is the event-list shown on the left side of the
   event editor.  It shows a list of numbered events, one per line.
   The currently-selected event is highlighted in cyan text on a black
   background.  Here is an example of the data line for a MIDI event:

   \begin{verbatim}
      17-003:3:128 Note On   Chan 3    Key 66 Vel 107
   \end{verbatim}

   This line consists of the following parts:

   \begin{enumber}
      \item \textbf{Index Number}
      \item \textbf{Time Stamp}
      \item \textbf{Event Name}
      \item \textbf{Channel Number}
      \item \textbf{Data Bytes}
   \end{enumber}

   \setcounter{ItemCounter}{0}      % Reset the ItemCounter for this list.

   \itempar{Index Number}{event editor!index number}
   Displays the index number of the event.
   This number is purely for the reference of the user, and is not part
   of the event.  Events in the pattern are numbered from 0 to the number of
   events in the pattern.  They serve as a way to better know where one is in
   the sequence.

   \itempar{Time Stamp}{event editor!time stamp}
   Displays the time stamp of the event.
   This value indicates the cumulative time of the event in the pattern.
   It is displayed in the format of "measure:beat:divisions".
   The measure values start from 1, and range up to the number of measures in
   the pattern.
   The beat values start from 1, and range up to the number of beats in the
   measure.
   The division values range from 0 up to one less than the
   \index{ppqn}
   PPQN (pulses per quarter note) value for the whole song.

   \itempar{Event Name}{event editor!event name}
   Displays the name of the event.
   The event name indicates what kind of MIDI event it is. 
   The following event names are supported:

   \begin{enumber}
      \item \textbf{Note Off}
      \item \textbf{Note On}
      \item \textbf{Aftertouch}
      \item \textbf{Control Change}
      \item \textbf{Program Change}
      \item \textbf{Channel Pressure}
      \item \textbf{Pitch Wheel}
   \end{enumber}

   \textbf{Note that these are all MIDI \textsl{channel events}.
   Support for MIDI \textsl{system events} is in place, but is not
   ready for exposure to the user.}

   \itempar{Channel Number}{event editor!channel number}
   Shows the channel number (for channel-events only).
   Be sure to note that, for the user, MIDI channels always range from
   1 to 16.  (Internally, they range from 0 to 15).

   \itempar{Data Bytes}{event editor!data bytes}
   Shows the one or two data bytes for the event.

   Note Off, Note On, and Aftertouch events requires a byte for the key (0 to
   127) and a byte for the velocity (also 0 to 127).
   Control Change events require a control code and a value for that control
   code.  Pitch wheel events require two bytes to encode the full range of
   pitch changes.

   Program change events require only a byte value to pick the patch or program
   (instrument) to be used for the sequence.  The Channel Pressure event
   requires only a one-byte value.

\subsubsection{Event Frame / Navigation}
\label{subsec:seq64_event_frame_navigation}

   Moving about in the event frame is fairly straightforward, but has some
   wrinkles to note.  (It was more difficult to get working than expected!)

   Navigation with the mouse is done by moving to the desired event and
   clicking on it.  The event becomes highlighted, and its data items are shown
   in the "info panel" (discussed in the next section).
   There is currently no support for dragging and dropping events in the event
   frame.

   The scrollbar can be used to move within the frame, either by one line at a
   time, or by a page at a time.  A page is defined as one frame's worth of
   lines, minus 5 lines, for some overlap in paging.

   Navigation with keystrokes is also supported, for the Up and Down arrows and
   the Page-Up and Page-Down keys.  Note that using the Up and Down arrows by
   holding them down for awhile causes autorepeat to kick in, and the updates
   become very erratic and annoying.  Use the scrollbar or page keys to
   move through multiple pages.

   \textbf{Bug:}
   \index{bugs!home/end keys}
   We still need to implement the Home and End keys!

\subsection{Event Editor / Info Panel}
\label{subsec:seq64_event_editor_info}

   The "info panel" is simply a read-only list of properties on the top right
   of the event editor.  It serves to remind the used of the sequence being
   edited and some characteristics of the sequence and the whole song.
   Currently, five items are shown:

   \begin{enumber}
      \item \textbf{Sequence Name}.
         This item is redundant, as the window caption for the event editor
         also shows the sequence name.  It can be set in the pattern editor.
      \item \textbf{Time Signature}.
         This item is a sequence property.  It can be set in the pattern
         editor.
      \item \textbf{PPQN}
         This item shows the "parts per quarter note", or resolution of the
         whole song.  The default PPQN of \textsl{Sequencer64} is 192.
      \item \textbf{Sequence Channel}
         In \textsl{Sequencer64}, the channel number is a property of the
         sequence.  All channel events in the sequence get routed to the same
         channel, even if somehow the event itself specifies a different
         channel.
      \item \textbf{Sequence Count}
         Displays the current number of events in the sequence.
         This number changes as events are inserted or deleted.
   \end{enumber}

\subsection{Event Editor / Edit Fields}
\label{subsec:seq64_event_editor_fields}

   The edit fields show the values of the currently-selected event.  They allow
   changing an event, adding a new event, or deleting the currently-selected
   event.

   \begin{enumber}
      \item \textbf{Event Category} (read-only)
      \item \textbf{Event Timestamp}
      \item \textbf{Event Name}
      \item \textbf{Data Byte 1}
      \item \textbf{Data Byte 2}
      \item \textbf{Delete Current Event}
      \item \textbf{Insert New Event}
      \item \textbf{Modify Current Event}
   \end{enumber}

   \setcounter{ItemCounter}{0}      % Reset the ItemCounter for this list.

   \itempar{Event Category}{event editor!event category}
   Displays the event category of the event.  Currently, only channel events
   can be handled, but someday we hope to handle the wide array of system
   events, and perhap even system-exclusive events.

   \itempar{Event Timestamp}{event editor!event timestamp}
   Displays the timestamp of the event.  Currently only the
   "measure:beat:division" format is fully supported.
   We allow editing (but not display) of the timestamp in
   pulse (divisions) format and "hour:minute:second.fraction" format, but
   there are bugs to work out.

   If one wants to delete or modify an event, this field does not need to be
   modified.  If this field is modified, and the \textbf{Modify Current Event}
   button is pressed, then the event will be moved.  This field can also locate
   a new event at a specific time.  If the time is not in the current frame,
   the frame will move to the location of the current event.

   \textbf{Bug:}
   \index{bugs!event timestamp change}
   Although the frame will move to the location of a new event, the location of
   a timestamp-modified event will not change.  Not sure that the delete/insert
   sequence works at all.

   \itempar{Event Name}{event editor!event name}
   Displays the name of the event, and allows entry of an event name.
   The event name indicates what kind of MIDI event it is. 
   The following event names are supported:

   \begin{enumber}
      \item \textbf{Note Off}
      \item \textbf{Note On}
      \item \textbf{Aftertouch}
      \item \textbf{Control Change}
      \item \textbf{Program Change}
      \item \textbf{Channel Pressure}
      \item \textbf{Pitch Wheel}
   \end{enumber}

   Typing in one of these names should change the kind of event if the event is
   modified.  Abbreviations and case-insensitivity can be used to reduce the
   effort of typing.

   \textbf{Bug:}
   \index{bugs!event name change}
   Currently, the handling of the editing of the event name is severely broken!
   Also, it would be better to provide a drop-down list for more painless
   selection of events.

   \itempar{Data Byte 1}{event editor!data byte 1}
   Allows the modification of the first data byte of the event.
   One must know what one is doing.
   The scanning of the digits is very simple:  start with the first digit, and
   convert until a non-digit is encountered.  The data-byte value can be
   entered in decimal notation, or, if prepended with "0x", in hexadecimal
   notation.

   \itempar{Data Byte 2}{event editor!data byte 2}
   Allows the modification of the second data byte of the event (if applicable
   to the event).
   One must know what one is doing.
   The scanning of the digits is very simple:  start with the first digit, and
   convert until a non-digit is encountered.  The data-byte value can be
   entered in decimal notation, or, if prepended with "0x", in hexadecimal
   notation.

\subsection{Event Editor / Bottom Buttons}
\label{subsec:seq64_event_editor_buttons}

%-------------------------------------------------------------------------------
% vim: ts=3 sw=3 et ft=tex
%-------------------------------------------------------------------------------
