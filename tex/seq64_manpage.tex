%-------------------------------------------------------------------------------
% seq64_manpage
%-------------------------------------------------------------------------------
%
% \file        seq64_manpage.tex
% \library     Documents
% \author      Chris Ahlstrom
% \date        2015-08-31
% \update      2018-04-29
% \version     $Revision$
% \license     $XPC_GPL_LICENSE$
%
%     Provides the man page section of seq24-user-manual.tex.
%
%-------------------------------------------------------------------------------

\section{Sequencer64 Man Page}
\label{sec:seq64_man_page}

   This section presents the contents of the \textsl{Sequencer64} man page, but
   not exactly in \textsl{man} format.  Also, an item or two are shown that
   somehow didn't make it into the man page, and minor corrections and
   formatting tweaks were made.
   For example, we replaced the underscore with the hyphen in the names of some
   options.  The legacy Seq24 options, which use underscores or are missing the
   option hyphen, are still unofficially supported.

   \texttt{\$HOME/.config/sequencer64/sequencer64.rc} holds the "rc" settings
   for \textsl{Sequencer64}.

   \texttt{\$HOME/.config/sequencer64/sequencer64.usr} holds the "user" settings
   for \textsl{Sequencer64}.

   But the old style names are used for the "legacy" mode.  See the
   \texttt{--legacy} option below. Here is the basic command line:

%  \textsl{Sequencer64} is a real-time MIDI sequencer. It was created to
%  provide a very simple interface for editing and playing MIDI 'loops'.

   \begin{verbatim}
       sequencer64 [OPTIONS] [FILENAME]
       seq64 [OPTIONS] [FILENAME]
   \end{verbatim}

   \textsl{Sequencer64} accepts the following options, plus an optional name of
   a MIDI file.  Please note that many of the options are 'sticky'.  If they
   are used on the command-line, their settings are saved to the configuration
   files when \textsl{Sequencer64} exits.

   \setcounter{ItemCounter}{0}      % Reset the ItemCounter for this list.

   \optionpar{-h}{--help}
      Display a list of all command-line options, then exit.

   \optionpar{-v}{--version}
      Display the program version, then exit.

   \optionpar{-H}{--home [directory]}
      \textbf{New:}
      \index{new!home directory}
      Change the "home" directory from \texttt{.contrib/sequencer64}
      (always relative to \texttt{\$HOME}.
      This option causes the \texttt{sequencer64.rc}
      and \texttt{sequencer64.usr} files to be loaded from or
      saved to a different directory.
      Format: \texttt{--home dirname}.

   \optionpar{-l}{--legacy}
      \textbf{New:}
      \index{new!legacy mode}
      Save the MIDI file in the old Seq24 format, as unspecified
      binary data, instead of as a legal MIDI track with meta events.
      Also read the configuration, if provided, from the "legacy"
      \texttt{\textasciitilde/.seq24rc} and
      \texttt{\textasciitilde/.seq24usr} files.

%     instead of the new
%     \texttt{\textasciitilde/.config/sequencer64/sequencer64.rc} and
%     \texttt{\textasciitilde/.config/sequencer64/sequencer64.usr} files.

      The user-interface will indicate this mode with a small text
      note.
      This mode is also used if \textsl{Sequencer64} is invoked as the
      \texttt{seq24} command (one can create a soft link to the sequencer64
      executable to make that happen).

   \optionpar{-b}{--bus [buss]}
      \textbf{New:}
      \index{new!MIDI buss override}
      Supports modifying the buss number on all tracks when a MIDI file is
      read.  All tracks are loaded with this buss-number override.  This
      feature is useful for testing, making it easy to map the MIDI file onto
      the system's current hardware/software synthesizer setup.
      Also note that this option applies the MIDI buss override to any new
      sequences, as well.

      Format: \texttt{--bus bussnumber} or
      \texttt{--buss bussnumber}.

      Most of the time, one will want to set this value to -1.

   \optionpar{-q}{--ppqn [ppqn]}
      \textbf{New:}
      \index{new!ppqn override}
      Supports modifying the PPQN value of Sequencer64, which is currently
      hardwired to a value of 192.  This setting should allow MIDI files to
      play back at the proper speed, and be written with the new PPQN value.
      This feature is basically done.
      It seems to work -- one can load MIDI files of arbitrary PPQN, and they
      play normally and look normal in the editor windows.  They can also be
      saved, and load with the new PPQN value. 
      But use the feature carefully for now, and save your work.
      We'll have more to say on this topic in the future.
      Format: \texttt{--ppqn ppqnnumber}.

   \optionpar{-L}{--lash}
      \textbf{New:}
      \index{new!LASH runtime enabling}
      If LASH support is compiled into the program, this option
      enables it.
      If LASH support is not compiled into the program, this option will not
      be shown in the output of the --help option.

   \optionpar{-n}{--no-lash}
      \textbf{New:}
      \index{new!LASH runtime disabling}
      If LASH support is compiled into the program, this option
      disables it, even if the default or configuration file set it.
      If LASH support is not compiled into the program, this option will not
      be shown in the output of the --help option.

   \optionpar{N/A}{--file [filename]}
      Load a MIDI file on startup.
      \textbf{Bug:}
      \index{bugs!--file option doesn't exist}
      This option does not exist.
      Instead, specify the file itself as the last command-line argument.

   \optionpar{-m}{--manual-alsa-ports}
      \textsl{Sequencer64} won't attach the system's existing ALSA ports.
      Instead, it will create is own set of input and output busses/ports.

   \optionpar{-a}{--auto-alsa-ports}
      \textsl{Sequencer64} will attach the system's existing ALSA ports.
      This variant is useful for overriding the rc configuration file.

   \optionpar{-r}{--reveal-alsa-ports}
      \textbf{New:}
      \index{new!reveal ALSA ports}
      \textsl{Sequencer64} will show the names of the ALSA port that the system
      defines, rather than the names defined in the 'user' configuration file.

   \optionpar{-R}{--hide-alsa-ports}
      \index{new!hide ALSA ports}
      \textsl{Sequencer64} will show the names of the ALSA port that the 'user'
      configuration file define, rather than the names defined by ALSA.

   \optionpar{-A}{--alsa}
      \textsl{Sequencer64} will not run the JACK support, even if specified
      in the configuration file.  The configuration options are sticky (they
      are saved), and sometimes they aren't what you want to run.

   \optionpar{-s}{--show-midi}
      Dumps incoming MIDI to the screen.

   \optionpar{-p}{--priority}
      Runs at higher priority with a FIFO scheduler.

   \optionpar{N/A}{--pass-sysex}
      Passes any incoming SYSEX messages to all outputs.
		Not yet supported.

   \optionpar{-i}{--ignore [number]}
      Ignore ALSA device [number].

   \optionpar{-k}{--show-keys}
      Prints pressed key value.

   \optionpar{-K}{--inverse}
      \index{inverse colors}
      \index{night mode}
      Changes the color scheme for the sequence editor and performance editor
      piano rolls.  It basicially inverts the colors.  It can be considered
      kind of a "night mode".

   \optionpar{-x}{--interaction-method [number]}
      Select the mouse interaction method.
      0 = seq24 (the default); and 1 = fruity loops method.
      The latter does not completely support all actions supported by the Seq24
      interaction method, at this time.

      The following options will not be shown by --help if the application is
      not compiled for JACK support.

   \optionpar{-j}{--jack-transport}
      \textsl{Sequencer64} will sync to JACK transport.

   \optionpar{-J}{--jack-master}
      \textsl{Sequencer64} will try to be JACK master.

   \optionpar{-C}{--jack-master-cond}
      JACK master will fail if there is already a master.

   \optionpar{-M}{--jack-start-mode [x]}
      When \textsl{Sequencer64} is synced to JACK, the following play modes
      are available: 0 = live mode; and 1 = song mode, the default.

   \optionpar{-S}{--stats}
      Print statistics on the command-line while running.
      Not available unless this option has been compiled in at build time,
      which can be determined by using the \texttt{--version} option.

   \optionpar{-U}{--jack-session-uuid [uuid]}
      Set the UUID for the JACK session.

   \optionpar{-u}{--user-save}
      Save the "user" configuration file when exiting Sequencer64.
      Normally, it is saved only if not present in the configuration directory,
      so as not to get stuck with temporary settings such as the --bus option.
      Note that the "rc" configuration option are generally also saved.
      But see the "auto-option-save" directive in the "rc" file.
      It is new with version 0.9.9.15.

   \optionpar{-f}{--rc filename}
      Use a different "rc" configuration file.  It must be a file in the user's
      \$HOME/.config/sequencer64 directory or the directory specified by the
      --home option.  Not supported by the --legacy mode.  The '.rc' extension
      is added if no extension is present in the filename.

   \optionpar{-F}{--usr filename}
      Use a different "usr" configuration file.  It must be a file in the
      user's \$HOME/.config/sequencer64 directory or the directory specified by
      the --home option.  Not supported by the --legacy mode.  The '.usr'
      extension is added if no extension is present in the filename.

   \optionpar{-c}{--config basename}
      Use a different configuration file base name for the 'rc' and 'usr'
      files.  For example, one can specify a full configuration for "testing",
      for "jack", or for "alsa", to set up
      \texttt{testing.rc} and \texttt{testing.usr},
      \texttt{jack.rc} and \texttt{jack.usr},
      \texttt{alsa.rc} and \texttt{alsa.usr}.

   \optionpar{-o}{--option opvalue}
      Provides additional options, since the application is running out of
      single-character options.  The \texttt{opvalue} set supported is:
      \begin{itemize}
         \item \texttt{daemonize}.
            \index{daemonize}
            Makes the \texttt{seq64cli} application
            fork to the background.
         \item \texttt{no-daemonize}.
            \index{no-daemonize}
            Makes the \texttt{seq64cli} application
            run in the foreground, where it is easy to see the informational
            output written to the console.
         \item \texttt{log=filename.log}.
            \index{log}
            Reroutes standard error and standard
            output messages to the given log-file.  This file is located in the
            directory for the "rc" and "usr" files (which can be altered via
            the \texttt{-H}/\texttt{--home} directory option).  If this file is
            already present, additional log information is appended to it.  If
            the "=filename.log" portion is missing, the log-file name in the
            \texttt{[user-options]} section of the "usr" file is used as the
            default log-file, if this file-name is not empty.
         \item \texttt{wid=3x2,i}.
            \index{multi-wid}
            Provides for multiple main windows ("mainwids") to be shown in a
            big grid, so that multiple sets can be viewed at the same time.
            The default is to show the usual single mainwid.
            The first number specified is the row count, which can range from 1
            to 3.  The second number specified is the column count, which can
            only be 1 or 2.  The third parameter starts with 't' ("true")
            to indicate that the multiple sets will be controlled by a single
            set spinner, which is the default.  Specifying 'f' ("false") or 'i'
            ("indep") will show one set spinner for each mainwid.  Note that it
            is possible, in this mode, to show the same set in two different
            mainwids, but this is not recommended, as there are minor
            unavoidable issues with that.  Also note that the format for this
            command line option is very strict, no deviations or added spaces.
            Finally, to save these options to the "usr" file, add the
            \texttt{--user-save} option to the command line.
            In that file, the options modified are \texttt{block\_rows} and
            \texttt{block\_columns}.
         \item \texttt{sets=8x8}.
            \index{variset}
            This option, informally known as "variset", allow some changes in
            the set size and layout from the default 4x8 = 32 sets layout.
            \textbf{Warning:}
            \textsl{seq24} was fairly hardwired for supporting 32 patterns per
            set, and there are still places where that is true.  Thus,
            consider this option to be experimental.
            To save these options to the "usr" file, add the
            \texttt{--user-save} option to the command line.
            In that file, the options modified are \texttt{mainwnd\_rows} and
            \texttt{mainwnd\_cols}.
         \item \texttt{scale=x.y}.
            \index{scaling}
            This option scales the main window by the factor x.y, which can
            range from 0.75 to 3.0.  This makes the main window larger, and
            most of its contents larger.  It does not currently scale the
            fonts used in the display.  A scale factor of 2.5 is about the
            maximum useful value.  If multi-wid is active, then you probably
            need a larger monitor to use this factor.
      \end{itemize}

   \texttt{\$HOME/.config/sequencer64.rc} holds the main configuration settings
   for all versions of Sequencer64.  If it does not exist, it will be generated
   when Sequencer64 exits.  If it does exist, it will be rewritten with the
   current configuration of Sequencer64.  Many, or most, of the command-line
   options are "sticky", in that they will be written to the configuration
   file.

   \texttt{\$HOME/.config/sequencer64.usr} stores the MIDI-configuration
   settings and some of the user-interface settings for Sequencer64.  If it
   does not exist, it will be generated with a minimal configuration when
   Sequencer64 exits.  If it does exist, it will be rewritten with the current
   configuration of Sequencer64, but \textsl{only} if the
   \texttt{--user-save} option was provided on the command-line.
   Note that the
   \texttt{--legacy} option causes the old
   configuration-file names (\texttt{.seq64rc} and \texttt{.seq64usr}
   in the \texttt{\$HOME} directory)
   to be used.

   The current Sequencer64 project homepage is a git repository at

   \url{https://github.com/ahlstromcj/sequencer64.git}.

   Up-to-date instructions can be found in the project at

   \url{https://github.com/ahlstromcj/sequencer64-doc.git}.

   The old Seq24 project homepage is at
   \url{http://www.filter24.org/seq24/} the new
   one is at \url{https://edge.launchpad.net/seq24/}.
   It is released under the GNU GPL license.
   Sequencer64 is also released under the GNU GPL license.

   \textsl{Sequencer64} was written by Chris Ahlstrom
   \href{mailto:ahlstromcj@gmail.com}{ahlstromcj@gmail.com}
   (with a fair amount of help).
   \textsl{Seq24} was written by Rob C. Buse
   \href{mailto:seq24@filter24.org}{seq24@filter24.org}
   and the \textsl{Sequencer64} team.

   This manual page was written by Dana Olson
   \url{mailto:seq24@ubuntustudio.com} with additions from Guido Scholz
   \url{mailto:guido.scholz@bayernline.de} and Chris Ahlstrom
   \url{mailto:ahlstromcj@gmail.com}.

%   \begin{verbatim}
% Version 0.94.0                      June 6 2017                  sequencer64(1)
%   \end{verbatim}

%-------------------------------------------------------------------------------
% vim: ts=3 sw=3 et ft=tex
%-------------------------------------------------------------------------------
