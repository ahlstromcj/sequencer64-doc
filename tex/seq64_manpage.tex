%-------------------------------------------------------------------------------
% seq64_manpage
%-------------------------------------------------------------------------------
%
% \file        seq64_manpage.tex
% \library     Documents
% \author      Chris Ahlstrom
% \date        2015-08-31
% \update      2015-11-21
% \version     $Revision$
% \license     $XPC_GPL_LICENSE$
%
%     Provides the man page section of seq24-user-manual.tex.
%
%-------------------------------------------------------------------------------

\section{Sequencer64 Man Page}
\label{sec:seq64_man_page}

   This section presents the contents of the \textsl{Sequencer64} man page, but
   not exactly in \textsl{man} format.  Also, an item or two are shown that
   somehow didn't make it into the man page, and minor corrections and
   formatting tweaks were made.
   For example, we replaced the underscore with the hyphen in the names of some
   options.  The legacy Seq24 options, which use underscores or are missing the
   option hyphen, are still unofficially supported.

   \texttt{\$HOME/.config/sequencer64/sequencer64.rc} holds the "rc" settings
   for \textsl{Sequencer64}.
   \texttt{\$HOME/.config/sequencer64/sequencer64.usr} holds the "user" settings
   for \textsl{Sequencer64}.
   But the old style names are used for the "legacy" mode.  See the
   \texttt{--legacy} option below.

   \textsl{Sequencer64} is a real-time MIDI sequencer. It was created to
   provide a very simple interface for editing and playing MIDI 'loops'.

   \begin{verbatim}
       sequencer64 [OPTIONS] [FILENAME]
   \end{verbatim}

   \textsl{Sequencer64} accepts the following options, plus an optional name of
   a MIDI file.

   \setcounter{ItemCounter}{0}      % Reset the ItemCounter for this list.

   \optionpar{-h}{--help}
      Display a list of all command-line options.

   \optionpar{-v}{--version}
      Display the program version.

   \optionpar{-l}{--legacy}
      \textbf{New:}
      \index{new!legacy mode}
      Save the MIDI file in the old Seq24 format, as unspecified
      binary data, instead of as a legal MIDI track with meta events.
      Also read the configuration, if provided, from the
      \texttt{~/.seq24rc} and \texttt{~/.seq24usr} files,
      instead of the new
      \texttt{~/.config/sequencer64/sequencer64.rc} and
      \texttt{~/.config/sequencer64/sequencer64.usr} files.
      The user-interface will indicate this mode with a small text
      note.
      This mode is also used if \textsl{Sequencer64} is invoked as the
      \texttt{seq24} command (one can create a soft link to the sequencer64
      executable to make that happen).

   \optionpar{-b}{--bus}
      \textbf{New:}
      \index{new!buss number override}
      Supports modifying the buss number on all tracks when a MIDI file is
      read.  All tracks are loaded with this buss-number override.  This
      feature is useful for testing, making it easy to map the MIDI file onto
      the system's current hardware/software synthesizer setup.

   \optionpar{-q}{--ppqn}
      \textbf{New:}
      \index{new!ppqn override}
      Supports modifying the PPQN value of Sequencer64, which is currently
      hardwired to a value of 192.  This setting should allow MIDI files to
      play back at the proper speed, and be written with the new PPQN value.
      This feature is \textbf{STILL IN PROGRESS}, and undependable.

   \optionpar{-L}{--lash}
      \textbf{New:}
      \index{new!LASH runtime enabling}
      If LASH support is compiled into the program, this option
      enables it.
      If LASH support is not compiled into the program, this option will not
      be shown in the output of the --help option.

   \optionpar{-n}{--no-lash}
      \textbf{New:}
      \index{new!LASH runtime disabling}
      If LASH support is compiled into the program, this option
      disables it, even if the default or configuration file set it.
      If LASH support is not compiled into the program, this option will not
      be shown in the output of the --help option.

   \optionpar{N/A}{--file [filename]}
      Load a MIDI file on startup.
      \textbf{Bug:}
      \index{bugs!--file option doesn't exist}
      This option does not exist.
      Instead, specify the file itself as the last command-line argument.

   \optionpar{-m}{--manual-alsa-ports}
      \textsl{Sequencer64} won't attach ALSA ports.
      Instead, it will create is own set of input and output busses/ports.

   \optionpar{-a}{--auto-alsa-ports}
      \textsl{Sequencer64} will attach ALSA ports.
      This variant is useful for overriding the rc configuration file.

   \optionpar{-s}{--show-midi}
      Dumps incoming MIDI to the screen.

   \optionpar{-p}{--priority}
      Runs at higher priority with a FIFO scheduler.

   \optionpar{N/A}{--pass-sysex}
      Passes any incoming SYSEX messages to all outputs.

   \optionpar{-i}{--ignore [number]}
      Ignore ALSA device [number].

   \optionpar{-k}{--show-keys}
      Prints pressed key value.

   \optionpar{-x}{--interaction-method [number]}
      Select the mouse interaction method.
      0 = seq24 (the default); and 1 = fruity loops method.
      The latter does not completely support all actions supported by the Seq24
      interaction method, at this time.

      The following options will not be shown by --help if the application is
      not compiled for JACK support.

   \optionpar{-j}{--jack-transport}
      \textsl{Sequencer64} will sync to JACK transport.

   \optionpar{-J}{--jack-master}
      \textsl{Sequencer64} will try to be JACK master.

   \optionpar{-C}{--jack-master-cond}
      JACK master will fail if there is already a master.

   \optionpar{-M}{--jack-start-mode [x]}
      When \textsl{Sequencer64} is synced to JACK, the following play modes
      are available: 0 = live mode; and 1 = song mode, the default.

   \optionpar{-S}{--stats}
      Print statistics on the command-line while running.

   \optionpar{-U}{--jack-session-uuid [uuid]}
      Set the UUID for the JACK session.

   \optionpar{-u}{--user-save}
      Save the "user" configuration file when exiting Sequencer64.
      Normally, it is saved only if not present in the configuration directory,
      so as not to get stuck with temporary settings such as the --bus option.

   The current Sequencer64 project homepage is a simple git repository at the
   https://github.com/ahlstromcj/sequencer64.git URL.
   Up-to-date instructions can be found in the project at the
   https://github.com/ahlstromcj/sequencer64-doc.git URL.

   The old Seq24 project homepage is at
   \url{http://www.filter24.org/seq24/} the new
   one is at \url{https://edge.launchpad.net/seq24/}.
   It is released under the GNU GPL license.

   Sequencer64 is also released under the GNU GPL license.

   \textsl{Sequencer64} was written by Chris Ahlstrom <ahlstromcj@gmail.com>.
   \textsl{Seq24} was written by Rob C. Buse \url{mailto:seq24@filter24.org}
   and the \textsl{Sequencer64} team.

   This manual page was written by Dana Olson
   \url{mailto:seq24@ubuntustudio.com} with additions from Guido Scholz
   \url{mailto:guido.scholz@bayernline.de} and Chris Ahlstrom
   \url{mailto:ahlstromcj@gmail.com}.

   \begin{verbatim}
Version 0.9.9.6                 November 4 2015                  sequencer64(1)
   \end{verbatim}

%-------------------------------------------------------------------------------
% vim: ts=3 sw=3 et ft=tex
%-------------------------------------------------------------------------------
