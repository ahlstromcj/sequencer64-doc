%-------------------------------------------------------------------------------
% seq64_manpage
%-------------------------------------------------------------------------------
%
% \file        seq64_manpage.tex
% \library     Documents
% \author      Chris Ahlstrom
% \date        2015-08-31
% \update      2015-09-05
% \version     $Revision$
% \license     $XPC_GPL_LICENSE$
%
%     Provides the man page section of seq24-user-manual.tex.
%
%-------------------------------------------------------------------------------

\section{Sequencer64 Man Page}
\label{sec:seq64_man_page}

   This section presents the contents of the \textsl{Sequencer64} man page, but
   not exactly in \textsl{man} format.  Also, an item or two are shown that
   somehow didn't make it into the man page, and minor corrections and
   formatting tweaks were made.

   \textsl{Sequencer64} is a real-time MIDI sequencer. It was created to
   provide a very simple interface for editing and playing MIDI 'loops'.

   \begin{verbatim}
       sequencer64 [OPTIONS] [FILENAME]
   \end{verbatim}

   \textsl{Sequencer64} accepts the following options, plus an optional name of a
   MIDI file.

   \setcounter{ItemCounter}{0}      % Reset the ItemCounter for this list.

   \optionpar{-h}{--help}
      Display a list of all command-line options.

   \optionpar{-v}{--version}
      Display the program version.

   \optionpar{-l}{--legacy}
      \textbf{New:}
      \index{new!legacy mode}
      Save the MIDI file in the old Seq24 format, as unspecified
      binary data, instead of as a legal MIDI track with meta events.
      Also read the configuration, if provided, from the
      \texttt{~/.seq24rc} and \texttt{~/.seq24usr} files,
      instead of the new
      \texttt{~/.config/sequencer64/sequencer64rc} and
      \texttt{~/.config/sequencer64/sequencer64usr} files.
      The user-interface will indicate this mode with a small text
      note.
      This mode is also used if \textsl{Sequencer64} is invoked as the
      \texttt{seq24} command (one can create a soft link to the sequencer64
      binary to make that happen).

   \optionpar{-L}{--lash}
      \textbf{New:}
      \index{new!LASH runtime enabling}
      If LASH support is compiled into the program, this option
      enables it.

   \optionpar{N/A}{--file [filename]}
      Load a MIDI file on startup.
      \textbf{Bug:}
      \index{bugs!--file option doesn't exist}
      This option does not exist.
      Instead, specify the file itself as the last command-line argument.

   \optionpar{-m}{--manual\_alsa\_ports}
      \textsl{Sequencer64} won't attach ALSA ports.

   \optionpar{-s}{--showmidi}
      Dumps incoming MIDI to the screen.

   \optionpar{-p}{--priority}
      Runs at higher priority with FIFO scheduler.

   \optionpar{N/A}{--pass\_sysex}
      Passes any incoming SYSEX messages to all outputs.

   \optionpar{-i}{--ignore [number]}
      Ignore ALSA device [number].

   \optionpar{-k}{--show\_keys}
      Prints pressed key value.

   \optionpar{-x}{--interaction\_method [number]}
      Select the mouse interaction method.
      0 = seq24 (the default); and 1 = fruity loops method.

   \optionpar{-j}{--jack\_transport}
      \textsl{Sequencer64} will sync to JACK transport.

   \optionpar{-J}{--jack\_master}
      \textsl{Sequencer64} will try to be JACK master.

   \optionpar{-C}{--jack\_master\_cond}
      JACK master will fail if there is already a master.

   \optionpar{-M}{--jack\_start\_mode [x]}
      When \textsl{Sequencer64} is synced to JACK, the following play modes
      are available: 0 = live mode; and 1 = song mode, the default.

   \optionpar{-S}{--stats}
      Print statistics on the command-line while running.

   \optionpar{-U}{--jack\_session\_uuid [uuid]}
      Set the UUID for the JACK session.

   \texttt{\$HOME/.seq24rc} holds the user settings for \textsl{Sequencer64}.

   The old project homepage is at
   \url{http://www.filter24.org/seq24/} the new
   one is at \url{https://edge.launchpad.net/seq24/}.
   It is released under the GNU GPL license.

   \textsl{Sequencer64} was written by Chris Ahlstrom <ahlstromcj@gmail.com>.
   \textsl{Seq24} was written by Rob C. Buse \url{mailto:seq24@filter24.org}
   and the \textsl{Sequencer64} team.

   This manual page was written by Dana Olson
   \url{mailto:seq24@ubuntustudio.com} with additions from Guido Scholz
   \url{mailto:guido.scholz@bayernline.de} and Chris Ahlstrom
   \url{mailto:ahlstromcj@gmail.com}.

   \begin{verbatim}
Version 0.9.3                   September 1 2015                  Sequencer64(1)
   \end{verbatim}

%-------------------------------------------------------------------------------
% vim: ts=3 sw=3 et ft=tex
%-------------------------------------------------------------------------------
